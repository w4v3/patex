%% PaTeX support for e-TeX extensions, almost identical to etex.src
%% assuming preloaded patex.tex
%% assuming INITEX

\message{Setting up e-TeX}
\rootme

% Define the basic error-reporting and abort mechanisms:

\def \et@xmsg #1#2%
   {\begingroup
    \def \n {^^J}%
    \def \ { }%
    \newlinechar=\expandafter `\n
    \if E#1%
          \errorcontextlines=0
          \errmessage {e-TeX error: #2}%
    \else
          \message {\n ! e-TeX \if I#1 message%
                               \else \if W#1 warning%
                                     \else \if F#1 fatal error%
                                           \else
                                               \ unknown (#1)%
                                           \fi
                                     \fi
                               \fi: #2%
                   }%
    \fi
    \endgroup
   }

% Note: a future version may report errors in the %<fac>-<s>-<code>[, <text>]
% format, allowing the more verbose <text>s to be read from file rather than
% stored in the format.

\def \et@xabort #1%
    {\et@xmsg {F}{#1}%
     \batchmode
     \end
    }

% Make sure this file is being read by e-TeX in extended mode;
% If it is, prepare to check version/revision compatibility, otherwise abort.

\ifx \undefined \eTeXversion
     \et@xabort {this file can be processed only in extended mode;\n
                 \ \ did you perhaps forget the asterisk?%
                }%
\else
     \begingroup
     \catcode `\%=12
     \catcode `\?=14
     \xdef \et@xfilehdr
          {\detokenize {%% e-TeX V}\the \eTeXversion \eTeXrevision}?
     \xdef \et@xlibhdr
          {\detokenize {%% e-TeXlib V}\the \eTeXversion \eTeXrevision}?
     \endgroup
\fi

% Assume extended mode, but no additional \catcodes/\defs yet;
% set up a simple e-TeX condition-reporting system:

\def \et@xinf #1{\et@xmsg {I}{#1}} %%% not currently used
\def \et@xwarn #1{\et@xmsg {W}{#1}}
\def \et@xerr   #1#2{{\errhelp={#2}\et@xmsg {E}{#1}}}

% deactivate the processing of patterns and exceptions; these will be
% reinstated later, after the \uselanguage mechanism has been defined.

\let \et@xpatterns=\patterns

%%% PaTeX: is already preloaded, without \patterns

% Assume an extended Plain environment (i.e. there are no longer any
% restrictions on the coding techniques we can use).  First prepare to
% carry out consistency checks on the file headers and the current e-TeX
% version/revision levels:

\newread \et@xinput

\def \etexhdrchk #1#2%
    {\openin \et@xinput=#2
     \ifeof \et@xinput
            \chardef \etexstatus=0 % V1.0-1
     \else
            \begingroup
            \endlinechar=-1
            \readline \et@xinput to \et@xbuf
            \closein \et@xinput
            \def \p@rtition ##1.##2\endp@rtition {##1}%            V2.1%0
            \xdef \et@xbuf
                {\expandafter \p@rtition \et@xbuf .\endp@rtition}% V2.1;0
            \xdef \et@xtmp {\csname et@x#1hdr\endcsname}%
            \xdef \et@xtmp
                {\expandafter \p@rtition \et@xtmp .\endp@rtition}% V2.1;0
            \endgroup
            \ifx \et@xtmp \et@xbuf
                 \chardef \etexstatus=2 % V1.1;4
            \else
                 \chardef \etexstatus=1 % V1.1;4
            \fi
     \fi
    }

% Check the version number of this file:

\def \et@xfmtsrc {etex.src}
\etexhdrchk {file} {\et@xfmtsrc}
\ifcase \etexstatus
        \et@xerr {unable to open format source file "\et@xfmtsrc";}
                   {This should not happen; please ensure that your system
                    allows a file to be opened for reading more than once
                    concurrently.%
                   }%
\or
        \et@xwarn {format source file "\et@xfmtsrc" has wrong header;\n
                    \ \ expected: "\et@xfilehdr"; found: "\et@xbuf";%
                  }%
                  {You are using a version of e-TeX which may be incompatible
                   with the source for the format file you are trying to
                   compile; please ensure that you have the most recent
                   version of each.  I will proceed, but you should treat
                   the results with caution...%
                  }%
\fi

% Module handling now implemented (V1.0-2)

\def \module #1{\iftrue}
\let \endmodule=\fi

\newtoks \et@xtoks

\def \et@xl@@d #1 #2\endl@ad %%% the "#1 #2" code avoids trailing spaces
    {\ifcsname module:#1\endcsname
             \et@xwarn {duplicate module name "#1"}%
     \else
        \csname module:#1\endcsname %%% we exploit the side-effect explicitly
        \et@xtoks=\expandafter
                {\the \et@xtoks
                 \expandafter \let \csname module:#1\endcsname=\undefined
                }%
     \fi
    }

\def \et@xl@ad #1#2,#3\endl@ad %%% the #1#2 code avoids spurious leading spaces
    {\et@xl@@d #1#2 \endl@ad
     \if *#3*
        \let \n@xt=\relax
     \else
        \def \n@xt {\et@xl@ad #3\endl@ad}%
     \fi
     \n@xt
    }

\def \et@xload #1 %%% needed because of the embedded \ifs...
    {\def \module ##1%
         {\unless
          \ifcsname module:##1\endcsname
                 \message {Skipping module "##1";}%
          \else
                 \message {Loading module "##1";}%
         }%
     \input #1
     \def \module ##1{\iftrue}%
    }

\def \load #1 from #2 %%% selective module loading from an e-TeX library file
    {\etexhdrchk {lib} {#2}%
     \ifcase \etexstatus
        \et@xerr {unable to open library file "#2"; load aborted.}
                   {I got an <eof> on trying to open your library file;
                    please make sure it exists, is readable and is not locked.%
                   }%
     \or
        \et@xerr {library file "#2" has wrong header;\n
                    \ \ expected: "\et@xlibhdr"; found: "\et@xbuf";\n
                    \ \ load aborted.%
                 }%
                 {The first line of your library does not start with
                  the correct e-TeX header; you may need to update the
                  file to be compatible with the current version of
                  e-TeX, or you may simply have specified the wrong file.%
                 }%
     \else
        \et@xtoks={\et@xtoks={}}%
        \et@xl@ad #1,\endl@ad
        \et@xload {#2}
        \the \et@xtoks
     \fi
    }

\load interactionmodes from etexdefs.lib %%% \load <foo>[, <bar>...] from <baz>

\def \@sk #1#2#3%
    {\ifnum \interactionmode=\interactionmodes {errorstop} %%% V2.0;15
        \def \pr@mpt {\csname #1 \endcsname}%
        \edef \pr@mptloop {{\escapechar=-1 \global \readline 16 to \pr@mpt}}%
        \loop \pr@mptloop
           \ch@ckforyn \pr@mpt {#2}%
        \ifb@dresponse
           \message {Please answer Y[es] or N[o]}%
        \repeat
     \else
        \def \pr@mpt {#3}%
        \ch@ckforyn \pr@mpt {#2}%
     \fi
    }

\def \ch@ckforyn #1#2% Bernd Raichle's improved version, V1.0-1
    {\edef \@nswer {#1}%
     \def \p@rse ##1##2\endp@rse
         {\lowercase {\if y##1}\b@dresponsefalse \csname #2true\endcsname
          \else \lowercase {\if n##1}\b@dresponsefalse \csname #2false\endcsname
                \else
                      \b@dresponsetrue
                \fi
          \fi
         }%
     \expandafter \p@rse \@nswer \endp@rse
    }

\def \usef@llback %%% V1.0-3
    {\message {Using fallback mode (USenglish)}%
     \addlanguage {USenglish}{hyphen}{}{2}{3}%
    }

% OK, that's all the utilities defined; on with the real work:
% First, re-instate \patterns:

\let \patterns=\et@xpatterns

% Define the language-handling commands

\def \et@xlang {\csname newlanguage\endcsname}

\def \uselanguage #1%
    {\ifcsname lang@#1\endcsname
        \language=\csname lang@#1\endcsname
        \lefthyphenmin=\csname lhm@#1\endcsname
        \righthyphenmin=\csname rhm@#1\endcsname
        \ifdefined \uselanguage@hook % V1.0-1
            \uselanguage@hook {#1}%  % V1.0-1
        \fi
     \else
        \et@xerr {language #1 undefined.}%
                 {You are trying to use a language which has not previously
                  been defined; remember that any language you want to use
                  will need to have been specified at the time the format
                  was created.%
                 }%
     \fi
    }

\def \addlanguage #1#2#3#4#5% language patterns exceptions lhm rhm
    {\expandafter \et@xlang \csname lang@#1\endcsname
     \expandafter \chardef \csname lhm@#1\endcsname=#4 % V1.0-1
     \expandafter \chardef \csname rhm@#1\endcsname=#5 % V1.0-1
     \uselanguage {#1}%
     \input #2
     \if *#3*\else \input #3 \fi
     \ifdefined \addlanguage@hook % V1.0-4
        \addlanguage@hook {#1}%  % V1.0-4
     \fi
    }

% Decrement \count 19, because \newlanguage will increment it again

\advance \count 19 by -1

% We are about to try to process a user/site-specific file "language.def",
% which establishes a Babel-like language selection environment.  Since
% there is always a risk of a spurious file of that name being found,
% we look for an e-TeX header in the first line.  If the file can be opened,
% but doesn't have the right header, we interrogate the user as to whether
% to use fallback mode; in this mode, we simply establish USenglish as the
% sole language, with the et@x patterns, exceptions and left- and right-
% hyphen minima for TeX.  If we can't interrogate the user (e.g. not in
% \errorstopmode), or if the user elects not to use fallback, we abort;
% if the file can't be found/opened, we use fallback unconditionally.

\newif \ifb@dresponse
\newif \ifusef@llback

\def \l@ngdefnfile {language.def}

\etexhdrchk {file} {\l@ngdefnfile}
\ifcase \etexstatus
        \et@xwarn {unable to open file "\l@ngdefnfile";}%
        \usef@llback                      % unable to open "language.def"
\or
        \et@xwarn {file "\l@ngdefnfile" has wrong header;\n
                      \ \ expected: "\et@xfilehdr"; found: "\et@xbuf";%
     		  }%
        \@sk {Use fallback?} {usef@llback} {y}%
        \ifusef@llback
                \usef@llback
        \else
                \input \l@ngdefnfile    % use "language.def" after warning
        \fi
\else
        \input \l@ngdefnfile    % "language.def" open & valid
\fi

% All that was just to set up natural language handling...
% The "real" work of "etex.src", however, is to augment the non-primitives
% of Plain.TeX to incorporate e-TeX specific features, and to add new
% non-primitives to simplify access to new e-TeX specific primitives.

\message {Augmenting the Plain TeX definitions:}
\message {\string \tracingall;}

\def \tracingall
    {\tracingonline=\@ne
     \tracingcommands=\thr@@ % plain.tex has \tw@
     \tracingstats=\tw@
     \tracingpages=\@ne
     \tracingoutput=\@ne
     \tracinglostchars=\tw@ % plain.tex has \@ne
     \tracingmacros=\tw@
     \tracingparagraphs=\@ne
     \tracingrestores=\@ne
     \showboxbreadth=\maxdimen
     \showboxdepth=\maxdimen
     \errorstopmode
     \tracinggroups=\@ne
     \tracingifs=\@ne
     \tracingscantokens=\@ne
     \tracingnesting=\@ne
     \tracingassigns=\tw@
     }

\message {Adding new e-TeX definitions:}

\message {\string \eTeX,}
\def \eTeX {$\varepsilon$-\TeX} %%% the simple version, not suitable for maths;
                                %%% a more sophisticated one may find its way
                                %%% into "etexdefs.lib" in due course.

\message {\string \loggingall,}
\def \loggingall {\tracingall \tracingonline=\z@}

\message {\string \tracingnone,}
\def \tracingnone
    {\tracingassigns=\z@
     \tracingnesting=\z@
     \tracingscantokens=\z@
     \tracingifs=\z@
     \tracinggroups=\z@
     \showboxdepth=\thr@@
     \showboxbreadth=5
     \tracingrestores=\z@
     \tracingparagraphs=\z@
     \tracingmacros=\z@
     \tracinglostchars=\@ne
     \tracingoutput=\z@
     \tracingpages=\z@
     \tracingstats=\z@
     \tracingcommands=\z@
     \tracingonline=\z@
    }

\message {register allocation;}

\newcount \et@xins % our insertion counter (\insc@unt is used differently)

% We have to adjust the Plain TeX register allocation counts for our
% slightly modified book-keeping:

\advance \count 10 by 1 % \count 10=23 % allocates \count registers 23, 24, ...
\advance \count 11 by 1 % \count 11=10 % allocates \dimen registers 10, 11, ...
\advance \count 12 by 1 % \count 12=10 % allocates \skip registers 10, 11, ...
\advance \count 13 by 1 % \count 13=10 % allocates \muskip registers 10, 11, ...
\advance \count 14 by 1 % \count 14=10 % allocates \box registers 10, 11, ...
\advance \count 15 by 1 % \count 15=10 % allocates \toks registers 10, 11, ...
\advance \count 16 by 1 % \count 16=0  % allocates input streams 0, 1, ...
\advance \count 17 by 1 % \count 17=0  % allocates output streams 0, 1, ...
\advance \count 18 by 1 % \count 18=4  % allocates math families 4, 5, ...
\advance \count 19 by 1 % \count 19=0  % allocates \language codes 0, 1, ...

\et@xins=\insc@unt % \et@xins=255 % allocates insertions 254, 253, ...

% We don't change the Plain TeX definitions of \newcount, etc., but the
% \alloc@ macro doing the actual work is redefined.

% When the normal register pool for \count, \dimen, \skip, \muskip,
% \box, or \toks registers is exhausted, we switch to the extended pool.

\def \alloc@ #1#2#3#4#5%
    {\ifnum \count 1#1 < #4% make sure there's still room
     	 \allocationnumber=\count 1#1
	 \global \advance \count 1#1 by \@ne
	 \global #3#5=\allocationnumber
	 \wlog {\string #5=\string #2\the \allocationnumber}%
     \else \ifnum #1 < 6
    		  \begingroup \escapechar=\m@ne
		  \expandafter \alloc@@ \expandafter {\string #2}#5%
	   \else	
	 	  \errmessage {No room for a new #2}%
	   \fi
     \fi
  }

% The \expandafter construction used here allows the generation of
% \newcount and \globcount from #1=count. Moreover (and more important)
% this construction avoids the appearance of \outer macros inside
% macro definitions or conditionals.

\def \alloc@@ #1#2%
    {\endgroup % restore \escapechar
     \message {Normal \csname#1\endcsname register pool exhausted,
						switching to extended pool.}%
     \global \expandafter 
     \let \csname new#1\expandafter \endcsname \csname glob#1\endcsname
     \csname new#1\endcsname#2%
    }

% We do change the Plain TeX definition of \newinsert

\outer \def \newinsert #1%make sure there's still room for ...
      {\ch@ck 0 \et@xins \count % ... a \count, ...
        {\ch@ck 1 \et@xins \dimen % ... \dimen, ...
     	  {\ch@ck 2 \et@xins \skip % ... \skip, ...
       	    {\ch@ck 4 \et@xins \box % ... and \box register
              {\global \advance \et@xins by \m@ne
	       \unless 
               \ifnum \insc@unt < \et@xins 
	      	      \global \insc@unt=\et@xins 
	       \fi
	       \allocationnumber=\et@xins
	       \global \chardef #1=\allocationnumber
	       \wlog {\string #1=\string \insert \the \allocationnumber}%
	      }%
            }%
          }%
        }%
      }

\def \ch@ck #1#2#3#4%
    {\ifnum \count 1#1 < #2#4\else \errmessage {No room for a new #3}\fi}

% And we define \reserveinserts, so that you can say \reserveinserts{17} 
% in order to reserve room for up to 17 additional insertion classes that will
% not be taken away by \newcount, \newdimen, \newskip, or \newbox.

\outer \def \reserveinserts#1%
      {\global \insc@unt=\numexpr \et@xins \ifnum #1 > \z@ -#1\fi \relax}

\message {extended register allocation;}

% Now, we define \globcount, \globbox, etc., so that you can say
% \globcount\foo and \foo will be defined (with \countdef) to be the
% next count register from the vastly larger but somewhat less efficient
% extended register pool. We also define \loccount, etc., but these
% register definitions are local to the current group.

\count 260=277     % globally allocates \count registers 277, 278, ...
\count 261=\@cclvi % globally allocates \dimen registers 256, 257, ...
\count 262=\@cclvi % globally allocates \skip registers 256, 257, ...
\count 263=\@cclvi % globally allocates \muskip registers 256, 257, ...
\count 264=\@cclvi % globally allocates \box registers 256, 257, ...
\count 265=\@cclvi % globally allocates \toks registers 256, 257, ...
\count 266=\@ne    % globally allocates \marks classes 1, 2, ...

\def \et@xmaxregs {32768}

\count 270=\et@xmaxregs % locally allocates \count registers 32767, 32766, ...
\count 271=\et@xmaxregs % ditto for \dimen registers
\count 272=\et@xmaxregs % ditto for \skip registers
\count 273=\et@xmaxregs % ditto for \muskip registers
\count 274=\et@xmaxregs % ditto for \box registers
\count 275=\et@xmaxregs % ditto for \toks registers
\count 276=\et@xmaxregs % ditto for \marks classes

% \count registers 256-259 and 267-269 are not (yet) used

\outer \def \globcount  {\et@xglob 0 \count  \countdef}
       \def \loccount   {\et@xloc  0 \count  \countdef}
\outer \def \globdimen  {\et@xglob 1 \dimen  \dimendef}
       \def \locdimen   {\et@xloc  1 \dimen  \dimendef}
\outer \def \globskip   {\et@xglob 2 \skip   \skipdef}
       \def \locskip    {\et@xloc  2 \skip   \skipdef}
\outer \def \globmuskip {\et@xglob 3 \muskip \muskipdef}
       \def \locmuskip  {\et@xloc  3 \muskip \muskipdef}
\outer \def \globbox    {\et@xglob 4 \box    \mathchardef}
       \def \locbox     {\et@xloc  4 \box    \mathchardef}
\outer \def \globtoks   {\et@xglob 5 \toks   \toksdef}
       \def \loctoks    {\et@xloc  5 \toks   \toksdef}
\outer \def \globmarks  {\et@xglob 6 \marks  \mathchardef}
       \def \locmarks   {\et@xloc  6 \marks  \mathchardef}

\let\newmark=\globmarks %%% 2.0;14
\let\newmarks=\globmarks

\def \et@xglob #1#2#3#4%
    {\et@xchk #1#2% make sure there's still room
    	{\allocationnumber=\count 26#1
	 \global \advance \count 26#1 by \@ne
	 \global #3#4=\allocationnumber
	 \wlog {\string #4=\string #2\the \allocationnumber}%
        }%
 }

\def \et@xloc#1#2#3#4%
    {\et@xchk #1#2% make sure there's still room
        {\advance \count 27#1 by \m@ne
         \allocationnumber=\count 27#1
 	 #3#4=\allocationnumber 
	 \wlog {\string #4=\string #2\the \allocationnumber \space (local)}%
       }%
 }

\def \et@xchk #1#2#3%
    {\ifnum \count 26#1 < \count 27#1 
        #3%
     \else 
	\errmessage {No room for a new #2}%
     \fi
    }

% Next we define \globcountblk, \loccountblk, etc., so that one can
% write \globcountblk\foo{17} and \foo will be defined (with \mathchardef)
% as the first (the zeroth?) of a block of 17 consecutive registers.
% Thus the user is intended to reference elements <\foo+0> to <\foo+n-1>,
% where n is the length of the block allocated.

\outer \def \globcountblk  {\et@xgblk 0 \count}
       \def \loccountblk   {\et@xlblk 0 \count}
\outer \def \globdimenblk  {\et@xgblk 1 \dimen}
       \def \locdimenblk   {\et@xlblk 1 \dimen}
\outer \def \globskipblk   {\et@xgblk 2 \skip}
       \def \locskipblk    {\et@xlblk 2 \skip}
\outer \def \globmuskipblk {\et@xgblk 3 \muskip}
       \def \locmuskipblk  {\et@xlblk 3 \muskip}
\outer \def \globboxblk    {\et@xgblk 4 \box}
       \def \locboxblk     {\et@xlblk 4 \box}
\outer \def \globtoksblk   {\et@xgblk 5 \toks}
       \def \loctoksblk    {\et@xlblk 5 \toks}
\outer \def \globmarksblk  {\et@xgblk 6 \marks}
       \def \locmarksblk   {\et@xlblk 6 \marks}

% And, both to provide a higher-level interface to the block allocation
% scheme, and to demonstrate possible applications of the new \...expr
% primitives, we also define \globcountvector and \loccountvector (etc)
% which allow the user to access elements with embedded arithmetic, as in

% \globcountvector \foo {12}
% \foo {\count 0 + 3} = \foo {\count 1 * 2}

\outer \def \globcountvector  {\et@xgvec 0 \count}
       \def \loccountvector   {\et@xlvec 0 \count}
\outer \def \globdimenvector  {\et@xgvec 1 \dimen}
       \def \locdimenvector   {\et@xlvec 1 \dimen}
\outer \def \globskipvector   {\et@xgvec 2 \skip}
       \def \locskipvector    {\et@xlvec 2 \skip}
\outer \def \globmuskipvector {\et@xgvec 3 \muskip}
       \def \locmuskipvector  {\et@xlvec 3 \muskip}
\outer \def \globboxvector    {\et@xgvec 4 \box}
       \def \locboxvector     {\et@xlvec 4 \box}
\outer \def \globtoksvector   {\et@xgvec 5 \toks}
       \def \loctoksvector    {\et@xlvec 5 \toks}
\outer \def \globmarksvector  {\et@xgvec 6 \marks}
       \def \locmarksvector   {\et@xlvec 6 \marks}

\def \et@xgblk #1#2#3#4%
    {\et@xchkblk #1#2{#4}% make sure there's still room
        {\allocationnumber=\count 26#1
         \global \advance \count 26#1 by #4%
         \global \mathchardef #3=\allocationnumber
         \wlog {\string #3=\string #2blk{\number #4} at \the \allocationnumber}%
      }%
    }

\def \et@xlblk #1#2#3#4%
    {\et@xchkblk #1#2{#4}% make sure there's still room
        {\advance \count 27#1 by -#4%
         \allocationnumber=\count 27#1
         \mathchardef #3=\allocationnumber
         \wlog {\string #3=\string #2blk{\number #4} 
       		at \the \allocationnumber \space (local)%
       	       }%
      }%
    }

\begingroup
\catcode `\# = 12
\gdef \et@xhash {#}
\endgroup

\def \et@xgvec #1#2#3#4%
    {\et@xchkblk #1#2{#4}% make sure there's still room
        {\allocationnumber=\count 26#1
         \global \advance \count 26#1 by #4%
         \ifx    #2\box
                 \def \2{}%
         \else   \ifx #2\marks
                      \def \2{}%
                 \else
                      \def \2{\string #2}%
                 \fi
         \fi
         \xdef #3##1{\2 \noexpand \numexpr \the \allocationnumber+##1\relax}%
         \wlog {\string #3 {\et@xhash 1} =
                  \2 {\the \allocationnumber+\et@xhash 1} (global #2 vector)%
      	     }%
      }%
    }

\def \et@xlvec #1#2#3#4%
    {\et@xchkblk #1#2{#4}% make sure there's still room
        {\advance \count 27#1 by -#4%
 	 \allocationnumber=\count27#1 
	 \mathchardef #3=\allocationnumber
	 \ifx  #2\box
               \def \2{}%
	 \else \ifx #2\marks
               	   \def \2{}%
               \else
               	   \def \2{\string #2}%
               \fi
	 \fi
	 \edef #3##1{\2 \noexpand \numexpr \the \allocationnumber+##1\relax}%
	 \wlog {\string #3 {\et@xhash 1} =
                   \2 {\the \allocationnumber+\et@xhash 1} (local #2 vector)%
               }%
        }%
    }

\def \et@xchkblk #1#2#3#4%
    {\ifnum #3 < \z@
     	   \errmessage {Negative register block size \number #3}%
     \else \ifnum \numexpr \count 26#1+#3 > \count 27#1
     		\errmessage {No room for new #2block of size \number#3}%
     	   \else 
	 	 #4%
     	   \fi 
     \fi
    }

% In an attempt to reduce the overheads of e-TeX, we recycle all possible
% resources, including (as a penultimate step) the recycler itself...
% The user can circumvent (or force) this during format creation,
% by \letting \ifrecycle=\iffalse (or \iftrue, to force it).

% As this file has grown by accretion, it is possible that we are no
% longer recycling all the resources we could; this will be investigated.

\def \mayber@cycle {\expandafter \ifrecycle}
\def \forcer@cycle {\expandafter \iftrue}

\ifdefined \ifrecycle
        \mayber@cycle
\else
        \forcer@cycle
\fi

% There's a concealed \if... lurking here, which explains the
% strange indentation of the code that follows (V1.0-1)

        \def \r@cycle #1%
            {\ifdefined #1
                 \message {\string #1,}\let #1=\und@fined
             \else
                 \message {\string #1 (not defined),}
             \fi
            }
        {\newlinechar=`\! \message {!Recycling:}}

        \r@cycle \addlanguage
        \r@cycle \@nswer
        \r@cycle \@sk
        \r@cycle \b@dresponsetrue
        \r@cycle \b@dresponsefalse
        \r@cycle \ch@ckforyn
        \r@cycle \mayber@cycle
        \r@cycle \et@xabort
        \r@cycle \et@xbuf
        \r@cycle \et@xfmtsrc
        \r@cycle \et@xfilehdr
        \r@cycle \et@xinf
        \r@cycle \et@xpatterns
%       \r@cycle \ifb@dresponse
%       \r@cycle \ifusef@llback
        \r@cycle \l@ngdefnfile
        \r@cycle \n@xt
        \r@cycle \p@rse
        \r@cycle \pr@mpt
        \r@cycle \pr@mptloop
        \r@cycle \forcer@cycle
        \r@cycle \usef@llback
        \r@cycle \usef@llbacktrue
        \r@cycle \usef@llbackfalse

% The following are retained, since they may be needed by user code; with a few
% (regrettable) exceptions, all of these are given the \etex or \et@x prefix,
% to reduce as far as possible the risk of them clashing with other used-defined
% names.

% The e-TeX team are willing to change the names of the remaining, at-risk,
% control sequences if it is demonstrated that the current names cause problems
% or difficulties.

        \def \r@tain #1%
            {\ifdefined #1
                 \message {\string #1,}
             \else
                 \message {\string #1 (not defined),}
             \fi
            }
        {\newlinechar=`\! \message {!Retaining:}}

        \r@tain \et@xerr
        \r@tain \et@xinput
        \r@tain \et@xlibhdr
        \r@tain \et@xmsg
        \r@tain \et@xtoks
        \r@tain \et@xwarn
        \r@tain \et@xl@@d
        \r@tain \et@xl@ad
        \r@tain \et@xload
        \r@tain \et@xlang
        \r@tain \et@xhash
        \r@tain \eTeX
        \r@tain \etexhdrchk
%       \r@tain \endmodule
        \r@tain \etexstatus
        \r@tain \module
        \r@tain \uselanguage

        \r@cycle \r@tain
        \r@cycle \r@cycle

\fi

% easier to remember way to active RTL typesetting
\def\activatertl{\TeXXeTstate=1}

\unrootme
